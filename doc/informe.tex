\documentclass[11pt,letterpaper]{article}
\usepackage[spanish]{babel}
%\usepackage[ansinew]{inputenc}
\usepackage[utf8]{inputenc}
% \usepackage[latin1]{inputenc}
\usepackage[letterpaper,includeheadfoot, top=0.5cm, bottom=3.0cm, right=2.0cm, left=2.0cm]{geometry}
\renewcommand{\familydefault}{\sfdefault}

\usepackage{graphicx}
\usepackage{color}
\usepackage{hyperref}
\usepackage{amssymb}
\usepackage{url}
%\usepackage{pdfpages}
\usepackage{fancyhdr}
\usepackage{hyperref}
\usepackage{subfig}

\usepackage{listings} %Codigo
\lstset{language=C, tabsize=4,framexleftmargin=5mm,breaklines=true}

\begin{document}
%\begin{sf}
% --------------- ---------PORTADA --------------------------------------------
\newpage
\pagestyle{fancy}
\fancyhf{}
%-------------------- CABECERA ---------------------
\fancyhead[L]{ \includegraphics[scale=0.9]{img/dcc.png} }
%------------------ TÍTULO -----------------------
\vspace*{6cm}
\begin{center}
\Huge  {Tarea 1}\\
\vspace{1cm}
\huge {Ordenamiento}\\
\vspace{1cm}
%\small {Título pe} \\
\end{center}
%----------------- NOMBRES ------------------------
\vfill
\begin{flushright}
\begin{tabular}{ll}
Autores: & Claudio Berroeta\\
& Sebastián Ferrada \\
Profesor: & Pablo Barceló\\
Auxiliar: & Miguel Romero\\
Ayudantes: & Javiera Born\\
& Giselle Font\\
& \today\\
& Santiago, Chile.
\end{tabular}
\end{flushright}

% ·············· ENCABEZADO - PIE DE PAGINA ············
\newpage
\pagestyle{fancy}
\fancyhf{}

%Encabezado
%\fancyhead[L]{\rightmark}
\fancyhead[L]{\small \rm \textit{Sección \rightmark}} %Izquierda
\fancyhead[R]{\small \rm \textbf{\thepage}} %Derecha


\fancyfoot[L]{\small \rm \textit{Pie de página - Izquierda}} %Izquierda
\fancyfoot[R]{\small \rm \textit{Pie de página - Derecha}} %Derecha
%\fancyfoot[C]{\thepage} %Centro

\renewcommand{\sectionmark}[1]{\markright{\thesection.\ #1}}
\renewcommand{\headrulewidth}{0.5pt}
\renewcommand{\footrulewidth}{0.5pt}

% =============== INDICE ===============

\tableofcontents
\listoffigures

% =============== SECCION ===============
\newpage
\section{Introducción}

% ----- Texto Introducción------
Blaaaaaaaaaaaaaaaaaaaaaaaaaaaaaaaaaaaaaaaaaaaaah.
% ----------------------------------

\newpage
\section{Algoritmos}

% ----- Texto Introductorio sobre los algoritmos------
Blaaaaaaaaaaaaaaaaaaaaaaaaaaaaaaaaaaaaaaaaaaaaah.
% -----------------------------------------------------------


\subsection{Bubble Sort}

% -----Explicación de cómo funciona Bubble Sort-------
Blaaaaaaaaaaaaaaaaaaaaaaaaaaaaaaaaaaaaaaaaaaaaah.
% -------------------------------------------------------------

\subsection{Insertion Sort}

% -----Explicación de cómo funciona Insertion Sort------
Blaaaaaaaaaaaaaaaaaaaaaaaaaaaaaaaaaaaaaaaaaaaaah.
% ---------------------------------------------------------------

\subsection{Merge Sort}

% -----Explicación de cómo funciona Merge Sort-------
Blaaaaaaaaaaaaaaaaaaaaaaaaaaaaaaaaaaaaaaaaaaaaah.
% -------------------------------------------------------------

\subsection{Quick Sort}

% -----Explicación de cómo funciona Quick Sort--------
Blaaaaaaaaaaaaaaaaaaaaaaaaaaaaaaaaaaaaaaaaaaaaah.
% -------------------------------------------------------------

\newpage
\section{Experimentos}
such experiment

\newpage
\section{Resultados}
very successful

\newpage
\section{Conclusión}
wow




% ============= ANEXOS =====================
\newpage
\section{Anexos}

\subsection{Algoritmos}
\subsubsection{Bubble Sort}
\subsubsection{Insertion Sort}
\subsubsection{Merge Sort}
\subsubsection{Quick Sort}

\subsection{Experimentos}
\subsubsection{Experimento 1}
\subsubsection{Experimento 2}
\subsubsection{Experimento 3}
\subsubsection{Experimento 4}

\subsection{Resultados}
\subsubsection{Resultado 1}
\subsubsection{Resultado 2}
\subsubsection{Resultado 3}
\subsubsection{Resultado 4}




% ============= FIN DE DOCUMENTO ==============
\end{document}

% % ················ IMAGEN ·················
% \begin{figure}[ht!]
% \centering
% \fbox{\includegraphics[scale=0.6]{img/torneo.png}}
% \caption{Torneo}\label{torneo}
% \end{figure}
% %··········································

% % ················ IMAGEN DOBLE ·················
% \begin{figure}[ht!] \centering
% \subfloat[Hola]{\includegraphics[scale=0.44]{img/holaquetal.png}}
% \subfloat[Que tal]{\includegraphics[scale=0.45]{img/holaquetal1.png}}
% \caption{Holaquetal}\label{holaquetal}
% \end{figure}
% %··········································